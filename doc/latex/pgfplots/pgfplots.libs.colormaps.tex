
\section{Colormaps}
\begingroup
\def\pgfplotsmanualcurlibrary{colormaps}

{\emph{An extension by Patrick Häcker}}


\begin{pgfplotslibrary}{colormaps}

A small library providing a number of additional |colormap|s. Many of these
|colormap|s originate from the free Matlab package ``SC -- powerful image
rendering'' of Oliver Woodford.

The purpose of this library is to provide further |colormap|s to all users and
to provide some of them which are similar to those used by
Matlab$^\text{\textregistered}$.

\begin{stylekey}{/pgfplots/colormap/autumn}
    A style which is equivalent to
    %
\begin{codeexample}[code only]
\pgfplotsset{
    /pgfplots/colormap={autumn}{rgb255=(255,0,0) rgb255=(255,255,0)}
}
\end{codeexample}
    \pgfplotsshowcolormap{autumn} \par \matlabcolormaptext
\end{stylekey}

\begin{stylekey}{/pgfplots/colormap/bled}
    A style which is equivalent to
    %
\begin{codeexample}[code only]
\pgfplotsset{
    /pgfplots/colormap={bled}{rgb255=(0,0,0) rgb255=(43,43,0) rgb255=(0,85,0)
        rgb255=(0,128,128) rgb255=(0,0,170) rgb255=(213,0,213) rgb255=(255,0,0)}
}
\end{codeexample}
    \pgfplotsshowcolormap{bled} \par \matlabcolormaptext
\end{stylekey}

\begin{stylekey}{/pgfplots/colormap/bright}
    A style which is equivalent to
    %
\begin{codeexample}[code only]
\pgfplotsset{
    /pgfplots/colormap={bright}{rgb255=(0,0,0) rgb255=(78,3,100) rgb255=(2,74,255)
        rgb255=(255,21,181) rgb255=(255,113,26) rgb255=(147,213,114) rgb255=(230,255,0)
        rgb255=(255,255,255)}
}
\end{codeexample}
    \pgfplotsshowcolormap{bright} \par \matlabcolormaptext
\end{stylekey}

\begin{stylekey}{/pgfplots/colormap/bone}
    A style which is equivalent to
    %
\begin{codeexample}[code only]
\pgfplotsset{
    /pgfplots/colormap={bone}{[1cm]rgb255(0cm)=(0,0,0) rgb255(3cm)=(84,84,116)
        rgb255(6cm)=(167,199,199) rgb255(8cm)=(255,255,255)}
}
\end{codeexample}
    \pgfplotsshowcolormap{bone} \par \matlabcolormaptext
\end{stylekey}

\begin{stylekey}{/pgfplots/colormap/cold}
    A style which is equivalent to
    %
\begin{codeexample}[code only]
\pgfplotsset{
    /pgfplots/colormap={cold}{rgb255=(0,0,0) rgb255=(0,0,255) rgb255=(0,255,255)
        rgb255=(255,255,255)}
}
\end{codeexample}
    \pgfplotsshowcolormap{cold} \par \matlabcolormaptext
\end{stylekey}

\begin{stylekey}{/pgfplots/colormap/copper}
    A style which is equivalent to
    %
\begin{codeexample}[code only]
\pgfplotsset{
    /pgfplots/colormap={copper}{[1cm]rgb255(0cm)=(0,0,0) rgb255(4cm)=(255,159,101)
        rgb255(5cm)=(255,199,127)}
}
\end{codeexample}
    \pgfplotsshowcolormap{copper} \par \matlabcolormaptext
\end{stylekey}

\begin{stylekey}{/pgfplots/colormap/copper2}
    A style which is equivalent to
    %
\begin{codeexample}[code only]
\pgfplotsset{
    /pgfplots/colormap={copper2}{rgb255=(0,0,0) rgb255=(68,62,63) rgb255=(170,112,95)
        rgb255=(207,194,138) rgb255=(255,255,255)}
}
\end{codeexample}
    \pgfplotsshowcolormap{copper2} \par \matlabcolormaptext
\end{stylekey}

\begin{stylekey}{/pgfplots/colormap/earth}
    A style which is equivalent to
    %
\begin{codeexample}[code only]
\pgfplotsset{
    /pgfplots/colormap={earth}{rgb255=(0,0,0) rgb255=(0,28,15) rgb255=(42,39,6)
        rgb255=(28,73,33) rgb255=(67,85,24) rgb255=(68,112,46) rgb255=(81,129,83)
        rgb255=(124,137,87) rgb255=(153,147,122) rgb255=(145,173,164) rgb255=(144,202,180)
        rgb255=(171,220,177) rgb255=(218,229,168) rgb255=(255,235,199) rgb255=(255,255,255)}
}
\end{codeexample}
    \pgfplotsshowcolormap{earth} \par \matlabcolormaptext
\end{stylekey}

\begin{stylekey}{/pgfplots/colormap/gray}
    A style which is equivalent to
    %
\begin{codeexample}[code only]
\pgfplotsset{
    /pgfplots/colormap={gray}{rgb255=(0,0,0) rgb255=(255,255,255)}
}
\end{codeexample}
    \pgfplotsshowcolormap{gray}

    This |colormap| is an alias for the standard |colormap/blackwhite|.

    \matlabcolormaptext
\end{stylekey}

\begin{stylekey}{/pgfplots/colormap/hot2}
    A style which is equivalent to
    %
\begin{codeexample}[code only]
\pgfplotsset{
    /pgfplots/colormap={hot2}{[1cm]rgb255(0cm)=(0,0,0) rgb255(3cm)=(255,0,0)
        rgb255(6cm)=(255,255,0) rgb255(8cm)=(255,255,255)}
}
\end{codeexample}
    \pgfplotsshowcolormap{hot2}

    Note that this particular choice ships directly with \PGFPlots{}, you do
    not need to load the |colormaps| library for this value.

    \matlabcolormaptext
\end{stylekey}

\begin{stylekey}{/pgfplots/colormap/hsv}
    A style which is equivalent to
    %
\begin{codeexample}[code only]
\pgfplotsset{
    /pgfplots/colormap={hsv}{rgb255=(255,0,0) rgb255=(255,255,0) rgb255=(0,255,0)
        rgb255=(0,255,255) rgb255=(0,0,255) rgb255=(255,0,255) rgb255=(255,0,0)}
}
\end{codeexample}
    \pgfplotsshowcolormap{hsv} \par \matlabcolormaptext
\end{stylekey}

\begin{stylekey}{/pgfplots/colormap/hsv2}
    A style which is equivalent to
    %
\begin{codeexample}[code only]
\pgfplotsset{
    /pgfplots/colormap={hsv2}{rgb255=(0,0,0) rgb255=(128,0,128) rgb255=(0,0,230)
        rgb255=(0,255,255) rgb255=(0,255,0) rgb255=(255,255,0) rgb255=(255,0,0)}
}
\end{codeexample}
    \pgfplotsshowcolormap{hsv2} \par \matlabcolormaptext
\end{stylekey}

\begin{stylekey}{/pgfplots/colormap/jet}
    A style which is equivalent to
    %
\begin{codeexample}[code only]
\pgfplotsset{
    /pgfplots/colormap={jet}{rgb255(0cm)=(0,0,128) rgb255(1cm)=(0,0,255)
        rgb255(3cm)=(0,255,255) rgb255(5cm)=(255,255,0) rgb255(7cm)=(255,0,0)
        rgb255(8cm)=(128,0,0)}
}
\end{codeexample}
    \pgfplotsshowcolormap{jet}

    Note that this particular choice ships directly with \PGFPlots{}, you do
    not need to load the |colormaps| library for this value.

    \matlabcolormaptext
\end{stylekey}

\begin{stylekey}{/pgfplots/colormap/pastel}
    A style which is equivalent to
    %
\begin{codeexample}[code only]
\pgfplotsset{
    /pgfplots/colormap={pastel}{rgb255=(0,0,0) rgb255=(120,0,5) rgb255=(0,91,172)
        rgb255=(215,35,217) rgb255=(120,172,78) rgb255=(255,176,24) rgb255=(230,255,0)
        rgb255=(255,255,255)}
}
\end{codeexample}
    \pgfplotsshowcolormap{pastel} \par \matlabcolormaptext
\end{stylekey}

\begin{stylekey}{/pgfplots/colormap/pink}
    A style which is equivalent to
    %
\begin{codeexample}[code only]
\pgfplotsset{
    /pgfplots/colormap={pink}{rgb255=(0,0,0) rgb255=(12,16,46) rgb255=(62,22,43)
        rgb255=(53,53,65) rgb255=(79,72,58) rgb255=(122,80,67) rgb255=(147,91,102)
        rgb255=(147,115,140) rgb255=(144,145,154) rgb255=(173,163,146) rgb255=(216,171,149)
        rgb255=(250,179,179) rgb255=(255,198,227) rgb255=(246,229,255) rgb255=(255,255,255)}
}
\end{codeexample}
    \pgfplotsshowcolormap{pink} \par \matlabcolormaptext
\end{stylekey}

\begin{stylekey}{/pgfplots/colormap/sepia}
    A style which is equivalent to
    %
\begin{codeexample}[code only]
\pgfplotsset{
    /pgfplots/colormap={sepia}{rgb255(0cm)=(0,0,0) rgb255(1cm)=(26,13,0)
        rgb255(18cm)=(255,230,204) rgb255(20cm)=(255,255,255)}
}
\end{codeexample}
    \pgfplotsshowcolormap{sepia} \par \matlabcolormaptext
\end{stylekey}

\begin{stylekey}{/pgfplots/colormap/spring}
    A style which is equivalent to
    %
\begin{codeexample}[code only]
\pgfplotsset{
    /pgfplots/colormap={spring}{rgb255=(255,0,255) rgb255=(255,255,0)}
}
\end{codeexample}
    \pgfplotsshowcolormap{spring} \par \matlabcolormaptext
\end{stylekey}

\begin{stylekey}{/pgfplots/colormap/summer}
    A style which is equivalent to
    %
\begin{codeexample}[code only]
\pgfplotsset{
    /pgfplots/colormap={summer}{rgb255=(0,128,102) rgb255=(255,255,102)}
}
\end{codeexample}
    \pgfplotsshowcolormap{summer} \par \matlabcolormaptext
\end{stylekey}

\begin{stylekey}{/pgfplots/colormap/temp}
    A style which is equivalent to
    %
\begin{codeexample}[code only]
\pgfplotsset{
    /pgfplots/colormap={temp}{rgb255=(36,0,217) rgb255=(25,29,247) rgb255=(41,87,255)
        rgb255=(61,135,255) rgb255=(87,176,255) rgb255=(117,211,255) rgb255=(153,235,255)
        rgb255=(189,249,255) rgb255=(235,255,255) rgb255=(255,255,235) rgb255=(255,242,189)
        rgb255=(255,214,153) rgb255=(255,172,117) rgb255=(255,120,87) rgb255=(255,61,61)
        rgb255=(247,40,54) rgb255=(217,22,48) rgb255=(166,0,33)}
}
\end{codeexample}
    \pgfplotsshowcolormap{temp} \par \matlabcolormaptext
\end{stylekey}

\begin{stylekey}{/pgfplots/colormap/thermal}
    A style which is equivalent to
    %
\begin{codeexample}[code only]
\pgfplotsset{
    /pgfplots/colormap={thermal}{rgb255=(0,0,0) rgb255=(77,0,179) rgb255=(255,51,0)
        rgb255=(255,255,0) rgb255=(255,255,255)}
}
\end{codeexample}
    \pgfplotsshowcolormap{thermal} \par \matlabcolormaptext
\end{stylekey}

\begin{stylekey}{/pgfplots/colormap/winter}
    A style which is equivalent to
    %
\begin{codeexample}[code only]
\pgfplotsset{
    /pgfplots/colormap={winter}{rgb255=(0,0,255) rgb255=(0,255,128)}
}
\end{codeexample}
    \pgfplotsshowcolormap{winter} \par \matlabcolormaptext
\end{stylekey}

\begin{stylekey}{/pgfplots/colormap/viridis high res}
    A style which installs the colormap ``viridis'' which has been defined by
    Stéfan van der Walt and Nathaniel Smith for |Matplotlib|. It is
    designated to be the default |colormap| for |Matplotlib| starting with
    version~$2.0$ and is released under the
    CC0\footnote{\url{https://creativecommons.org/about/cc0}}. This is the
    original suggestion as released by the authors.

    Please refer to \url{http://bids.github.io/colormap/} for details.
    %
\begin{codeexample}[code only]
\pgfplotsset{
    colormap name=viridis,
}
\end{codeexample}
    \pgfplotsshowcolormap{viridis}

\begin{codeexample}[code only]
\pgfplotsset{
    colormap/viridis high res,
}
\end{codeexample}
    \pgfplotsshowcolormap{viridis high res}
\end{stylekey}

\begin{stylekey}{/pgfplots/colormap/plasma high res}
    A style which installs the colormap ``plasma'' which like
    ``viridis'' (see above) has been defined by Stéfan van der Walt and
    Nathaniel Smith for |Matplotlib|. This is the original suggestion as
    released by the authors.

    Please refer to \url{http://bids.github.io/colormap/} for details.
    %
\begin{codeexample}[code only]
\pgfplotsset{
    colormap name=plasma,
}
\end{codeexample}
    \pgfplotsshowcolormap{plasma}

\begin{codeexample}[code only]
\pgfplotsset{
    colormap/plasma high res,
}
\end{codeexample}
    \pgfplotsshowcolormap{plasma high res}
\end{stylekey}

\begin{stylekey}{/pgfplots/colormap/gouldian high res}
    A style with installs the colormap ``gouldian'' (CET-L20) which has been
    designed by Peter Kovesi and is similar to the Parula colormap in
    Matlab$^\text{\textregistered}$. It is released under the
    \href{http://creativecommons.org/licenses/by/4.0/legalcode}{Creative Commons
    BY License}.

    Please refer to \url{https://colorcet.com/userguide/#gouldianparula} and
    \url{https://colorcet.com/download/index.html} for details.
    %
\begin{codeexample}[code only]
\pgfplotsset{
    colormap name=gouldian,
}
\end{codeexample}
    \pgfplotsshowcolormap{gouldian}

\begin{codeexample}[code only]
\pgfplotsset{
    colormap/gouldian high res,
}
\end{codeexample}
    \pgfplotsshowcolormap{gouldian high res}
\end{stylekey}

\end{pgfplotslibrary}
\endgroup
